% !iTexMac(input): thesis.tex

% Information about the thesis
\newcommand{\myTitle}{Compiling Functional Languages\xspace}
\newcommand{\myDegree}{Bachelor of Arts\xspace}
\newcommand{\myName}{Jeremy W. Sherman\xspace}
\newcommand{\myAdvisor}{Dr. Karsten Henckell\xspace}
\newcommand{\myCommittee}%
    {Dr. Karsten Henckell, Dr. Patrick McDonald, Dr. David Mullins\xspace}
\newcommand{\myDivision}{Division of Natural Sciences\xspace}
\newcommand{\myUni}{\protect{New College of Florida}\xspace}
\newcommand{\myLocation}{Sarasota, Fla.\xspace}
\newcommand{\myTime}{May 2008\xspace}

% Semantic markup
\newcommand{\code}[1]{\texttt{#1}} %the variable \code{x}
\newcommand{\foreign}[1]{\textit{#1}} %the \foreign{Entscheidungsproblem}
\newcommand{\vocab}[2][1]{\spacedlowsmallcaps{#2}} %the \vocab{lambda terms} are...
%\newcommand{\abbrev}[1]{\allcapsspacing{#1}} % introduced unwanted space before
\newcommand{\abbrev}[1]{#1} % \abbrev{IBM}
\newcommand{\Deutsch}[1]{\foreign{\begin{otherlanguage}{ngerman}#1\end{otherlanguage}}}
% Expressing substitution: replace:#1 with:#2 in:#3
\newcommand{\replace}[3]{\ensuremath{#3\left[#1 := #2\right]}} %M[x := N]
%\newcommand{\replace}[3]{\ensuremath{#3\left[#2 / #1\right]}} %M[N/x]

% Simple abbreviations
\newcommand{\lambdacalc}{lambda calculus\xspace} %We discuss the \lambdacalc...
\newcommand{\Lambdacalc}{Lambda calculus\xspace} %\Lambdacalc is really cool.
\newcommand{\TM}{Turing machine\xspace}
\newcommand{\TMs}{Turing machines\xspace}
\newcommand{\CISC}{\abbrev{CISC}\xspace}
\newcommand{\CISCs}{\abbrev{CISC}s\xspace}
\newcommand{\RISC}{\abbrev{RISC}\xspace}
\newcommand{\RISCs}{\abbrev{RISC}s\xspace}
\newcommand{\FA}{finite automaton\xspace}
\newcommand{\FAs}{finite automata\xspace}
\newcommand{\regex}{regular expression\xspace}
\newcommand{\regexes}{regular expressions\xspace}
\newcommand{\CFG}{context-free grammar\xspace}
\newcommand{\CFGs}{context-free grammars\xspace}
\newcommand{\PDA}{pushdown automaton\xspace}
\newcommand{\PDAs}{pushdown automata\xspace}

% More complex macros
% Two options for part-and-namerefs. First has problems if hyphenation interferes with the hyperref'd part and also creates too much emphasized text. Second emphasizes (via a link) only the part name.
%\newcommand{\partandnameref}[2]{\hyperref[#2]{#1}~\ref{#2}, \nameref{#2}}
\newcommand{\partandnameref}[2]{#1~\ref*{#2}, \nameref{#2}} % \partandnameref{Chapter}{label} -> Chapter 9, NAME

% Math: Semantic markup
\newcommand{\set}[1]{\ensuremath{\left\{ #1 \right\}}} % X = {1, 2, 3}
\newcommand{\union}{\ensuremath{\cup}}
\newcommand{\from}{\colon} % f: X -> Y
\newcommand{\emptyword}{\ensuremath{\epsilon}}
\newcommand{\kstar}{\ensuremath{^{\star}}{}}
\newcommand{\posclos}{\ensuremath{^{+}}{}}
\newcommand{\alt}{\ensuremath{+}}

% Math: Simple abbreviations
\let\oldprime=\prime
\renewcommand{\prime}{\ensuremath{^{\oldprime}}{}} % automatic superscript; random {} at the end is necessary to avoid complaints from LaTeX about double superscripts when one \prime directly follows another, e.g. v\prime\prime.