\partlevelChapter{Epilogue}\label{epilogue:epilogue}
The body of this work is through. This epilogue reflects on what you have just read about the functional and imperative families and offers some thoughts on future developments.

% Input major sections:
\section*{Looking Back}\label{conclusion:back}
\subsection*{Imperative and Functional Languages}
The models of computation that underlie both the imperative and functional families were developed around the same time, but the development of the von~Neumann machine set imperative languages on the path to ascendancy.

At first, these machines could be programmed only by manipulation of their hardware. The development of software brought assembly language, the prototypical imperative language, to the fore. Even today, assembly language cannot be beat for the control over the underlying machine it brings, but this control comes at great cost: in programming time, and in portability. It takes a long time to write a substantial amount of assembly code, and the code is then tied to the platform it was written for.

In the 1950s, Fortran brought imperative languages to a higher level of abstraction; later imperative languages brought more powerful abstractions. Still, early Fortran remains, though primitive, recognizably imperative. Within a decade, Lisp would be born. Lisp was an important predecessor for today's functional languages: Lisp made higher-order functions available, and so one faces similar problems compiling Lisp as compiling today's functional languages. But Lisp started as Lisp and continues as Lisp: there is no mistaking Lisp code for anything but Lisp code, and Lisp style is quite distinct from the style of modern functional programming.

It was another decade before \ML made its debut in the 1970s. It started as an interpreted language without the concept of algebraic data types, which was borrowed later from another language. The lazy branch would not begin to bear fruit until the 1980s. Over the next two decades, the functional language family would grow into its modern form.

In order to have any hope of displacing assembly as the dominant programming language, Fortran had to be fast, and it was designed from the outset with speed in mind. Lisp grew up with artificial intelligence and was adopted because it was very well-suited to programming in that domain. It competed on features and the power of its abstractions, not on speed. It pioneered garbage collection, but it took decades of research to get past the ``stop the world'' effect that scanning the heap and scavenging useful data can cause if done without sufficient sophistication. Since many application domains for programming languages demand speed, Lisp was only ever a marginal language outside symbolic processing. The imperative family would continue to look on garbage collection as an expensive and unneeded luxury until languages developed for object-oriented programming showed that it can bring new levels of productivity: computers are fast enough that writing, not executing, the code becomes the bottleneck in developing software.

\ML grew out of work on a theorem prover, and it too was developed (using Lisp, no less) to serve its application domain. Its type system could provide guarantees for theorem proving that a weaker system could not. Significant work was required to make both Lisp and \ML run decently fast on ``stock hardware.'' Partly for this reason, many persons researched alternative computing architectures meant to support such languages directly, just as the von~Neumann architecture naturally supports programs written in imperative languages. But stock hardware eventually won out, and it was only in the 1990s that optimizations were discovered to make lazy functional languages at all competitive with compiled imperative languages on stock hardware.

The bottom line, for all programming languages, is the machine they must eventually run on. 

problems of imp prog paradigm:
still not abstract enough?
backus's criticisms still valid: word-at-a-time, von~Neumann bottleneck

problems for fctl prog paradigm:
unusual
avoid backus's criticisms of anarchy/too much freedom through typing

\subsection*{Imperative and Functional Compilers}
this is reflection on the focuses of compilers for the two languages and what their problems are

problems for compiling imp:
pointers
data flow
do not handle recursion well
not so parallel-friendly

problems for compiling func:
closures
computational model mismatch
data structures
purity

work on both continues

\section*{Looking Forward}\label{conclusion:forward}
Functional languages are growing up. They are beginning to see increasing use in industry and increasing interest among programmers. They also hold out promise as a way to deal with the rise of ubiquitous symmetric multiprocessors, which brings the problems of concurrent programming out of scientific and network programming and into programming in general.

Functional languages also continue to influence imperative languages. Java brought garbage collection into the mainstream. Several imperative languages, including Microsoft's C\#, now allow anonymous functions; the programming language Python borrowed list comprehensions from Haskell; the spirit of declarative programming, if not explicitly functional programming, shows through in the language-integrated query (\abbrev{LINQ}) facilities added to the \abbrev{.NET} platform. Functional languages have been implemented for both the Java virtual machine (Scala) and Microsoft's \abbrev{.NET} platform (F\#, developed and promoted by Microsoft itself).

Functional programming, and declarative programming in general, appears to promise increased programmer productivity. As programming time continues to become the limiting factor in what is doable in software, this could lead to increasing adoption of functional languages. At the same time, imperative languages have begun to assume more elements of functional programming. It is possible that something like Objective Caml or Microsoft's F\# will become the dominant programming language in the next couple of decades.

These families are merging in some ways, but they also continue to develop in their own peculiar ways. Object-oriented programming has led to aspect-oriented programming; there is continuing research in the functional programming community on more powerful type systems, including those embracing dependent types and observational type theory. There are also attempts to extend functional languages in the direction of logic languages.

The families of programming languages continue to diversify, branch out, and join together. Their fortunes also change as new languages grow in popularity and old ones fall out. Old languages are sometimes made new again through a revised definition or new extensions that breathe life back into them. These are exciting times.

\section*{Suggestions for Further Research}
This thesis touched on a wide variety of subjects, from computer architecture to formal language theory and automata, parsing, optimizations, imperative and functional languages, the \lambdacalc and type theory. Each of these subjects has already had much said about it. References to related work have been given in the bibliographic notes at the end of each chapter.

If paradigms have captured your fancy, you might want to investigate a paradigm we did not have time to explore, logic programming. This paradigm is exemplified by the language Prolog. Work is ongoing to make constraint programming, an offshoot of logic programming, a viable paradigm. We mentioned attempts to blend functional and imperative features in a single language; there have also been attempts to create so-called functional logic languages, such as Curry.

Work to date on virtual machines for functional languages has approached them from a formal point of view. There is a significant body of literature treating virtual machines in themselves that explores optimizing and improving them. Applying this literature to the virtual machines used with functional languages could perhaps yield interesting results.

\section*{Bibliographic Notes}
Algebraic data types were borrowed by \ML from Burstall's Hope~\citep{Burstall:HOPE:1980}. We referred frequently to Backus's influential 1978 Turing award lecture~\citep{Backus:Can-programming:1978}.

Functional languages have seen significant use in industry. Development of Objective Caml is funded in part by a consortium including Intel, Microsoft, Jane Street Capital~\citep{Minsky:Caml:2008}, and LexiFi; the last two companies are involved in trading and finance. \Citet{Hudak:A-history:2007} give a list of companies using Haskell along with descriptions of how they use the language in addition to examples of the language's impact in higher education. Wadler\footnote{``Functional Programming in the Real World,'' \url{http://homepages.inf.ed.ac.uk/wadler/realworld/}.} maintains an extensive list of applications of functional programming. Appel\footnote{``Implementation work using \ML{},'' \url{http://www.cs.princeton.edu/~appel/smlnj/projects.html}} keeps up a smaller list of implementation work done using \ML{}. Wadler~\citep{Wadler:Why-no-one-uses:1998} provides an insightful analysis of why functional languages are not used more.

Observational type theory~\citep{Altenkirch:Observational:2007} is an interesting and powerful idea, while dependent types are powerful enough to express a variety of concepts that must otherwise be built into a language or done without~\citep{Altenkirch:Why-Dependent:2005,McKinna:Why-dependent:2006}. Meijer has worked to introduce concepts from functional programming into the imperative programming world, and he provides an excellent overview~\citep{Meijer:Confessions:2007} of this work.

\fixmarks