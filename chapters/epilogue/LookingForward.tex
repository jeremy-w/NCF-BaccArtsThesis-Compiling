\section*{Looking Forward}\label{conclusion:forward}
Functional languages are growing up. They are beginning to see increasing use in industry and increasing interest among programmers. They also hold out promise as a way to deal with the rise of ubiquitous symmetric multiprocessors, which brings the problems of concurrent programming out of scientific and network programming and into programming in general.

Functional languages also continue to influence imperative languages. Java brought garbage collection into the mainstream. Several imperative languages, including Microsoft's C\#, now allow anonymous functions; the programming language Python borrowed list comprehensions from Haskell; the spirit of declarative programming, if not explicitly functional programming, shows through in the language-integrated query (\abbrev{LINQ}) facilities added to the \abbrev{.NET} platform. Functional languages have been implemented for both the Java virtual machine (Scala) and Microsoft's \abbrev{.NET} platform (F\#, developed and promoted by Microsoft itself).

Functional programming, and declarative programming in general, appears to promise increased programmer productivity. As programming time continues to become the limiting factor in what is doable in software, this could lead to increasing adoption of functional languages. At the same time, imperative languages have begun to assume more elements of functional programming. It is possible that something like Objective Caml or Microsoft's F\# will become the dominant programming language in the next couple decades.

These families are merging in some ways, but they also continue to develop in their own peculiar ways. Object-oriented programming has led to aspect-oriented programming; there is continuing research in the functional programming community on more powerful type systems, including those embracing dependent types and observational type theory. There are also attempts to extend functional languages in the direction of logic languages.

The families of programming languages continue to diversify, branch out, and join together. Their fortunes also change as new languages grow in popularity and old ones fall out. Old languages are sometimes made new again through a revised definition or new extensions that breathe life back into them. These are exciting times.