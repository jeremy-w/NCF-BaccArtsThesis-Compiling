\myChapter{Conclusion}\label{background:conclusion}
This part provided background information essential to understanding the remainder of this work.
\begin{itemize}
\item In \partandnameref{Chapter}{background:beginnings}, we introduced the basic ideas of \lambdacalc and \TMs. These provide the fundamental models of computation for the functional and imperative paradigms, respectively. This connection will be made clearer in the following parts.

\item In \partandnameref{Chapter}{background:computers}, we used Turing machines as a bridge to modern computers. Succeeding sections described the three major parts of a computer: processor, memory, and input-output. Roughly, the processor is what lets a computer compute, memory provides storage, and input-output is what makes computers useful by allowing them to affect and interact with the world. We stressed the variety of processor architectures while giving some taste of that variety. We explained the existence of a memory hierarchy as well as the obstacle it presents to execution speed. We gave a rough sketch of how input-output is implemented in computers. We did not have much to say beyond this, since many of the details of input-output are more pertinent to programming languages themselves rather than their compilers.

\item In \partandnameref{Chapter}{background:compilers}, we surveyed compiler architecture and design. We introduced the three-part structure of a compiler and discussed each part. Along the way, we sketched the theory that lies at the basis of each part and how it is used in practice. We also briefly surveyed \IRs and their importance to the compiler. Lastly, we broached the chicken-and-egg issue of developing a compiler for a new programming language, implementing a compiler in its own source language, and similar compiler construction problems. The important point is that compilers neither develop in a vacuum nor spring fully-formed from the pregnant void, but evolve gradually, though this evolution may involve the seemingly contradictory device of the compiler effectively ``pulling itself up by its own bootstraps.''
\end{itemize}