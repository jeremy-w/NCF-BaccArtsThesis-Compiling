\section{Front End: Analyzing Source Code}\label{background:compilers:front}
The front end of the compiler is responsible for analyzing the source code. It takes a long string of characters (the program), discerns what each string is meant to be in terms of the ``types of speech'' of the programming language, figures out how the various parts fit together to form a valid program (or that they do not form a valid program, if the programmer has made an error!), and tries to infer the meaning of the parts. The first two steps are similar to the spell-checking and grammar-checking performed by a wordprocessor. The last step is one wordprocessors have not quite achieved just yet. An analogy along the same lines as the others, however, would be ``sense-checking'' or ``sanity-checking,'' which would answer the question, ``This is a syntactically valid sentence, yes, but does it mean anything, or is it nonsense?'' As a whole, the front end of a compiler represents one of the great achievements of computer science: we have powerful formalisms that can be used to specify and automatically generate it.

\subsection{Lexical Analysis}
As presented to the compiler, the source code is a very long sequence of characters. This is the domain of \vocab{lexical analysis}. A long sequence of characters does not mean much at the character-level, so the first thing the front end must do is proceed from characters to a more meaningful level of abstraction. The \vocab{lexer}, which performs lexical analysis (and is also called, quite naturally, the \vocab{lexical analyzer}), reads in characters and chunks them into \vocab{tokens}, strings of characters having some meaning at the level of the programming language's structure. These tokens are akin to parts of speech in spoken language\empause while the specific details of the token (``this identifier is formed by the string \code{engineIsRunning}'') might be recorded for use in later stages, they are subsumed by the token, which treats, in a sense, all nouns as nouns, regardless of whether one is ``cat'' and one is ``dog.''

This tokenization is performed systematically by simulating the operation of a \vocab{finite automaton} that recognizes tokens. A finite automaton is, like a \TM, an abstract machine, but it is far simpler and far less powerful: a \TM can do everything a \FA can, but a \FA cannot do everything a \TM can.

\subsubsection{Regular Languages}
It turns out that we can describe all decision problems as \vocab{language problems}. A language is a (potentially countably infinite) set of \vocab{words}, and words are made up of characters from a finite \vocab{alphabet} by \vocab{concatenation}, the ``chaining together'' of characters denoted by writing them without intervening space: concatenating $a$ and $b$ in that order gives $ab$. The decision problem recast as a language problem becomes, ``Given a word and a language (and, implicitly, an alphabet), determine whether the word is in the language.'' The languages for which a \TM can solve this problem are known variously as \vocab{recursive}, \vocab{decidable}, and \vocab{Turing-computable} languages. The languages whose membership problems can be solved by a \FA{}, on the other hand, are known as the \vocab{regular languages} and form a proper subset of the recursive languages.

\subsubsection{Finite Automata}
A \FA is a constructive way to describe a regular language. Each \FA is associated directly to a language, the language whose membership problem it solves. Given a word, it solves this problem by examining the word one character at a time. After it has consumed all its input, it halts operation. Based on the state in which it halts, we say either that it \vocab{accepts} the word or rejects it. We build a \FA by specifying its makeup. A \FA is made up of a finite set of states and a transition function that describes how, in each state, the \FA responds to consuming the characters of the alphabet. In specifying a \FA{}, we also specify the alphabet of its language, the \FA's initial state, and the set of \vocab[final state]{final} or \vocab{accepting states}, those states which, when the \FA halts in them, indicate acceptance of the word.

We can specify the states and transition function in two ways: either as a list and table (respectively), or graphically through a \vocab{transition diagram}. A transition diagram has circular nodes for states, typically labeled with the state name, and arrows between states, which indicate the transition function. The arrows are labeled with the character causing the state transition indicated by the arrow. Accepting states are indicated by circling the node representing their states, so that they appear as two concentric circles.

The form of the transition function distinguishes between several varieties of \FAs. A transition function that, on any character, permits a transition to only one state is known as a \vocab{deterministic \FA{}}. A transition function that permits a transition to a set of states on any character is known as a \vocab{non-deterministic \FA{}}. It accepts if any state out of the set of states it halts in is an accepting state. A final variety of \FA is distinguished by admitting not only transitions to a set of states, but ``autonomous'' transitions\empause transitions that occur without consuming any of the input. These are known as \vocab{\emptyword-transitions} because transitioning along them ``consumes'' only the empty word \emptyword\ made up of no characters. These varieties of \FAs are all equivalent in power\empause it is possible to convert a \FA of one type into another type such that both recognize the same language\empause but some sorts describe a language more naturally or concisely than others. \FAs[F] are unique in that, for a given regular language, there is a \vocab{minimal deterministic \FA{}}, a deterministic \FA with the fewest number of states possible that is unique up to renaming of states.

\subsubsection{Regular Expressions}
We can also describe regular languages declaratively, using \vocab{regular expressions}. These do not describe how to recognize a given language, but rather describe the language directly. This is done by augmenting the alphabet with a direct linguistic interpretation and by adding special symbols representing operations on this linguistic interpretation. 

The linguistic interpretation associated to a character is direct and intuitive: the character $a$ represents the language consisting of that single character, $\set{a}$. It is natural to generalize this direct representation to words: the word $w$ represents the language consisting of that single word, $\set{w}$. Words are built up by concatenation. To aid in describing many concatenations of a simple structure, we can introduce some notation. Iterated concatenation of a \regex $w$ with itself is represented by superscripts: $w^{0}$ is the language of only \emptyword, the empty word; $w^{1}$ is just $\set{w}$ itself; and, as a rule, $w^{n} = w^{n-1}w$. We can represent unlimited concatenation using the \vocab{Kleene star} $\kstar$: $w\kstar$ represents the set of all concatenations of the language represented by $w$ with itself, including $w^{0}$: $w\kstar = \set{w^{0}, w^{1}, w^{2}, \dotsc}$. If we wish to exclude the possibility of $w^{0},$ we can use the otherwise equivalent \vocab{positive closure} operator $\posclos$: $w\posclos = \set{w^{1}, w^{2}, \dotsc}.$ To represent choice or \vocab{alternation} in the language\empause either \regex $w$ or \regex $v$ is acceptable\empause we can introduce a corresponding operator; $+$ and $\vert$ are both popular choices for representing it: we shall use \alt\ here. Thus, the \regex $a \alt b$ represents the language $\set{a, b},$ while, more generally, the \regex $w \alt v$ constructed by the alternation of the \regexes $w$ and $v$ represents the language $L(w) \union L(v)$, where we use $L(w)$ to represent the language associated to the \regex $w$.\footnote{In general, where $X$ is any description of a language, whether by \TM or \FA or \regex or by any other description aside from the sets representing the languages themselves directly, we write $L(X)$ for the language described by $X$.} Finally, to allow unambiguous composition of \regexes, we can introduce clarifying parentheses. These let us describe, for example, the language $(a \alt b)\kstar b$, the language comprising all strings of zero or more $a$s or $b$s ending with a $b$.

While \regexes are very useful for describing regular languages, they do not provide a way to recognize the languages they describe. Fortunately, regular expressions happen to be readily interconvertible with \FAs.

\subsubsection{Lexers}
With \regexes to describe the lexical structure of tokens and \FAs to perform the actual work of recognizing tokens, we have a ready way to perform tokenization. Simply scan through the character stream till every recognizing \FA will begin to fail; of those that make it this far and will accept, select the token of the highest priority as that summing up the scanned text. The introduction of prioritization and a ``maximal munch'' rule provide ways to resolve ambiguity deriving from our wishing to chunk an input, the program, that in truth belongs to a language unrecognizable by a \FA{}, into words belonging to various token-languages recognized by \FAs.

In truth, however, the lexer is responsible for more than simply recognizing tokens. It works in cooperation with the parser (which we shall describe next) by feeding it a stream of tokens. Further, it records information associated to the tokens, often in a global, shared \vocab{symbol table} associating to each token, or symbol, some information, such as the text or value of the token. It might even use information in the symbol table or information provided by the parser to make a distinction between tokens that it is impossible or exceedingly difficult to make with \regexes alone.


\subsection{Syntax Analysis}
\vocab{Syntax analysis} follows lexical analysis. If lexical analysis is concerned with categorizing words by part of speech, then syntax analysis is concerned with understanding how these parts of speech are grammatically related and whether the sentences so formed are grammatical or not.

\subsubsection{Context-Free Languages}
In fact, ``grammatical'' is precisely the word, for the formalism affording ready syntax analysis is that of context-free grammars. As with the regular languages, we are able to describe a given context-free language either constructively or declaratively. The context-free languages are a proper superset of the regular languages and a proper subset of the recursive languages. Roughly, the context-free languages are distinguished from the regular languages by their ability to describe ``matching bracket'' constructs, such as the proper nesting of parentheses in an arithmetic expression, while the recursive languages are distinguished from the context-free languages in part by their ability to cope with context-sensitive languages.\footnote{The context-sensitive languages are, however, only a proper subset of the recursive languages.}

\subsubsection{Context-Free Grammars}
We use \vocab{context-free grammars} to specify context-free languages declaratively. As with \regexes and \FAs, context-free grammars operate in the context of a specific alphabet. The letters of the alphabet are called \vocab{terminals} or \vocab{terminal symbols}. \CFGs[C] augment this alphabet with a finite set of \vocab{non-terminals (non-terminal symbols)} to be used in specifying grammatical \vocab{productions}, which function as \vocab{rewrite rules}. Together, the set of terminal and non-terminal symbols are called \vocab{grammar symbols}, as they specify all the symbols used by the grammar. Analogous to the start state of the \FA is the context-free grammar's distinguished \vocab{start symbol}. Putting these rules together, one arrives at a grammar specification like the following:
\[
G = \left(N, T, \Sigma, P, S\right)
\]%
\[
N = \set{A, B} \qquad T = \set{a, b} \qquad \Sigma = {a, b}
\]
\[\begin{split}
P = \{&S \produces A, &&S \produces B, &&S \produces aABb,\\
&A \produces a \alt \emptyword, &&B \produces b \alt \emptyword\}
\end{split}\]
where $N$ is the set of non-terminals, $T$ the set of terminals, $\Sigma$ the alphabet, and $P$ the set of productions, where \produces\ is read as ``produces.'' The symbol to the left of the arrow is called the \vocab{head} of the production, while those to the right are called the \vocab{body}. %(The symbol \produces is also sometimes written \altproduces.)
``Computation'' proceeds by substitution: for example,
\[
S \derives[S \produces aABb] aABb \derives[A \produces a] aaBb \derives[B \produces b] aabb
\]
where \derives\ is read as ``derives in one step'' and the rule justifying the derivation is written above the arrow. Taking a cue from regular expressions, we can also write \derives[\star] for ``derives in zero or more steps'' (all grammar symbols derive themselves in zero steps) and \derives[+] for ``derives in one or more steps,'' where the productions justifying the derivation are implicit in the superscript star; the keen reader should perhaps like to construct their own explicit, step-by-step derivation. The language defined by the grammar is defined to be those strings made up only of terminal symbols that can be derived from the start symbol.

\paragraph{Parse Trees} We can use a \vocab{parse tree} to represent the derivation of a word in the language without concern for unnecessary sequencing of derivations imposed by our sequential presentation. For example, our choice to derive $a$ from $A$ prior to deriving $b$ from $B$ above is irrelevant, but that we first derived $aABb$ from $S$ before performing either of the remaining derivations is not, since the heads of these derivations are introduced by the derivation from $S$. We define parse trees constructively: 
\begin{aenumerate}
\item Begin by making the start symbol the root.
\item\label{parsetree:construction:choose} Select a non-terminal on the leaves of the tree with which to continue the derivation and a production for which it is the head.
\item Create new child nodes of the chosen head symbol, one for each symbol in the body.
\item Repeat from \ref{parsetree:construction:choose}.
\end{aenumerate}
At any point in time, the string of symbols derived thus far\empause those on the leaves, read in the same order applied to the child nodes in the body of a production\empause is called a \vocab{sentential form}. The process terminates when a word in the language is derived, as no non-terminal leaf nodes remain.
%FIXME: Demonstrate parse trees with TREES!

\paragraph{Ambiguity} Parse trees represent the derivation of a word without regard to unnecessary sequencing. A given tree represents a given parse. If more than one parse tree can derive the same word in the language, the grammar is said to be \vocab{ambiguous}. This corresponds to the use of a significantly different ordering of productions and potentially even of a different set of productions. The grammar is called ambiguous because, given such a word, it is uncertain which productions were used to derive it. The grammar we gave above is ambiguous when it comes to the empty word \emptyword, because $S \derives A \derives \emptyword$ and $S \derives B \derives \emptyword$ are both valid derivations with corresponding significantly different valid parse trees of \emptyword. However, if we were to eliminate the productions $S \derives A$ and $S \derives B$ from the grammar, we would then have an unambiguous grammar for the context-free language comprising $\set{ab, aab, abb, aabb} \equiv \set{a^{i}b^{j} \where 1 \leq i, j \leq 2}$. The only sequential choices are insignificant: in deriving $aabb$, we must have derived $aABb$ from $S$, but following that, did we first derive the second $a$ or the second $b$?

\paragraph{Derivation Order} While the parse trees for a word in a language factor out differences between possible derivations of the word other than those reflecting ambiguity in the grammar, when performing a derivation or constructing such a parse tree, we must employ such ``insignificant'' sequencing. There are two primary systematic ways to do so: always select the leftmost nonterminal symbol in step \ref{parsetree:construction:choose} of the parse tree construction process, or always select the rightmost. These ways of deterministically choosing the next symbol to replace in the derivation give rise to what are unsurprisingly known as a \vocab{leftmost derivation} and a \vocab{rightmost derivation}; to indicate the use of one or the other, the derivation arrow in all its forms is augmented with a subscript of $lm$ for leftmost derivation and $rm$ for rightmost, giving \derives[][lm] and \derives[][rm]. Since this is only a matter of choice in constructing the parse tree, it should be clear that, for any given parse tree, there exist both leftmost and rightmost derivations of its sentential form. %Leftmost and rightmost derivations will be important when we discuss parsing, which, given a word, must solve the problem of finding a valid derivation. (The problem of what to do if the sentence should prove ungrammatical, that is, if no valid derivation exists, is another problem in itself, that of error handling.)

\subsubsection{Pushdown Automata}
We can also specify context-free languages constructively using an abstract machine called a \vocab{pushdown automaton}. A \PDA is a \FA augmented with a stack and associated stack alphabet. It has an initial stack symbol as well as an initial state. Its transition function and behavior is complicated by its being inherently non-deterministic. As might be expected, its transition function is parameterized by the current input symbol, current state, and the symbol currently on top of the stack. However, for each such triple, the transition function specifies a set of pairs. Each pair consists of a state and a sequence of stack symbols with which to replace the current top of stack. For a given triple, the \PDA simultaneously transitions to all the states indicated by the transition function and replaces the symbol on top of the stack with the corresponding symbols for each new state it is in. Each one of these can be treated as a new \PDA. To ``move,'' each member of the family of \PDAs consults the current input, its state, and the top of the stack, and then transitions accordingly. We again have a choice of representing this either with a table or graphically. While \FA transition diagrams had arrows labelled %
%FIXME: The paragraph introducing PDAs likely needs more work. These things are kind of confusing to explain. Yay non-determinism...
\[
\token{input symbol}
\]
the arrows of \PDA transition diagrams are labelled 
\[
\token{input symbol}, \token{stack symbol to pop} / \token{stack symbols to push}
\]
where the convention used for the stack operations is that the symbol that is to be on top of the stack after pushing is leftmost (that is, the stack conceptually grows to the left).

There are some casualties of the transition from \FAs to the increased descriptive power of \PDAs. \PDAs are inherently non-deterministic: they always admit \emptyword-transitions and can be in a set of states at any given time. This non-determinism is essential for them to define the context-free languages. The languages described by deterministic \PDAs, while still a proper superset of the regular languages, are only a proper subset of the context-free languages. Further, there is no algorithmic way to produce a minimal \PDA for a given language. This poses a particular problem for parsing: as with lexing, we would like to use grammars to describe the syntactic structure and \PDAs to perform parsing by recognizing that structure, but we must now find some way for our inherently deterministic computers to cope with this inherent non-determinism in a reasonable amount of time.

\subsubsection{Parsers}
As exaggeratedly hinted at above, while grammars define a language, parsers are faced with an input that they must characterize as either of that language or not. They must, in fact, do more than simply check that their input is grammatical: they must construct an \IR of their input to pass on to the next part of the compiler.

We also mentioned the problem of the non-determinism inherent to \CFGs and \PDAs. So long as we only face insignificant questions of sequencing, we will have no problem determining what to do next. Realistic inputs do not require truly non-deterministic parsing. A program is meant to have a single meaning: to correspond to a parse tree, not a parse forest. Non-determinism occurs in parsing a programming language when the available context is insufficient to predict the shape of the parse tree, and it becomes necessary to entertain several possibilities simultaneously. Eventually, more context will be available to resolve the ambiguity, and we can return to building a single parse tree and abandon the others as false starts. Problems such as these are likely to affect only part of the input, and methods have been developed that handle such ``temporary non-determinism'' gracefully. %NOTE: Run this paragraph by Henckell again. It was a confusing one before.

The remainder of our discussion of parsers will focus on several of the more common of their many types. The level of our discussion will be one of summary, not of definition; for details, the interested reader is referred to the literature discussed in \partandnameref{Section}{background:conclusion:bibliographicnotes}.

\paragraph{Recursive Descent Parsers}
Recursive descent parsers discover a \emph{leftmost derivation} of the input string during a \emph{left-to-right scan} of the input, whose alphabet, thanks to the lexer, will be tokens rather than individual letters and symbols. One function is responsible for handling each token; parsing begins by calling the function associated to the start symbol. They discover the derivation by recursively calling themselves as necessary. The parser is aware of the current input symbol via what is known as \vocab{lookahead}. Since we are dealing with an actual machine, however, we are not restricted to lookahead of a single symbol, though we might prefer to do with only a single symbol's lookahead for effiency' sake. Those grammars parsable by a recursive descent parser with $k$ tokens of lookahead are known as \vocab{LL(k)}: leftmost derivation by left-to-right scan employing $k$ tokens of lookahead.

When recursive descent parsers use one token of lookahead, they act much like a \PDA. The implicitly managed function call stack acts as the \PDA's stack. However, since they trace out a leftmost derivation with only a limited number of tokens of lookahead, they must anticipate the proper derivation with minimal information about the rest of the input stream. This makes recursive descent parsers one of the most limited forms of parsers, though they might be the parser of choice in some cases because of the naturalness of expression they can admit and the simplicity and compactness of their parsers. Many of the disadvantages of recursive descent parsers can be overcome by admitting variable tokens of lookahead, with more tokens being used as needed to disambiguate the choice of production. 
%TODO: Mention ANTLR in the conclusion!

\paragraph{Precedence Parsing}
Recursive descent parsers are sometimes coupled with precedence parsers in order to facilitate parsing of arithmetic expressions. The order in which operations should be carried out is determined by a frequently implicit grouping determined by operator associativity and precedence. For example, multiplication is normally taken to have higher precedence than addition, so that $3 \times 5 + 4$ is understood to mean $(3 \times 5) + 4 = 19$ and not $3 \times (5 + 4) = 27.$ The left associativity of multiplication determines that $2 \times 2 \times 2$ should be understood as $(2 \times 2) \times 2.$ This becomes important, for example, in cases where the operands have side effects: suppose \code{id} is a unary function printing its input and then returning its input unchanged. Assume further that arguments to operators are evaluated left-to-right. Then \code{id}(1) + \code{id}(2) + \code{id}(3) will print \code{123} if addition is understood to be left-associative, but it will print \code{231} if addition is understood to be right-associative, even though the result of the additions will be identical due to the associativity property of addition.\footnote{If argument evaluation proceeded right-to-left, \code{213} and \code{321} would be printed instead.}

Operator precedence parsing is preferred over the use of LL$(k)$ grammar rules not only because it is somewhat unobvious how to enforce the desired associativity and precedence in an LL$(k)$ grammar, but also because doing so introduces a chain of productions that exist solely to enforce the desired associativity and precendence relations between the expression operators. Beyond its use in concert with recursive descent parsers, precedence parsing has mostly been subsumed by the class of grammars we shall describe next. The central idea of using precedence and associativity to disambiguate an otherwise ambiguous choice of productions has lived on in implementations of parser generators for this later class. Without recourse to a way other than grammatical productions to indicate precedence and associativity, grammars would often have to take a form that unnecessarily obscures their meaning simply to grammatically encode the desired precedence and associativity relations.

\paragraph{LR$(k)$ Parsers}
The \vocab[LR$(k)$ family of parsers]{LR$(k)$}\empause left-to-right scan, rightmost derivation with $k$ tokens of lookahead\empause family of parsers is perhaps the most commonly used in practice. I say ``family'' because a number of subtypes (to be discussed shortly) were developed to work around the exponential space and time requirements of the original \abbrev{LR}$(k)$ algorithm. The class of grammars recognizable by an \abbrev{LR}$(k)$ parser is known as the \abbrev{LR}$(k)$ grammars, and it is possible to give a reasonably straightforward \abbrev{LR}$(k)$ grammar for most programming languages. However, it was some time before clever algorithms that avoided unnecessary requirements of exponential space and time were developed, and so other, more restrictive classes of grammars with less demanding parsers were developed and deployed. Parser generators targeting these classes are more limited in terms of the grammars they can generate parsers for, not in terms of the languages such grammars can recognize: all parsers of the LR$(k)$ family, where $k > 0$, accept the same class of languages; they simply place different, more or less restrictive demands on the form of the grammars describing those languages. %NOTE: This para was confusing in the past. Run it by Henckell again.

Where LL$(k)$ parsers create a derivation from the top down by starting with the goal symbol and eventually building a derivation for the input, LR$(k)$ parsers build a rightmost derivation in reverse by reading in the input till they determine that they have seen the body of a production and then reducing the body to the head. They eventually reduce the entire input to the start symbol (often in this context called the \vocab{goal symbol}), at which point parsing is complete. They use a stack to store the symbols seen and recognized so far, so in the course of parsing they carry out a very limited set of actions: shifting input onto their stack, reducing part of the stack to a single symbol, accepting the input as a valid word in the grammar, and indicating an error when none of the above applies. Because of this behavior, such a bottom-up parser is often called a \vocab{shift-reduce parser}.

\paragraph{SLR$(k)$ Parsers}
The earliest and most restricted such class is known as the \vocab{simple LR$(k)$}, or \abbrev{SLR}$(k)$. These parsers use a simplistic method of determining what action to take while in a given state and reading a given input that introduces conflicts that more sophisticated methods would be capable of resolving. In a shift-reduce parser, there are two possible types of conflicts: 
\begin{description}
\item[\vocab{shift/reduce conflicts}] where the parser has seen what it considers the body of a valid production at this point in the parse but has also seen a viable prefix of yet another production, so it cannot determine whether to reduce using the former or shift further symbols onto the stack in an attempt to recognize the latter.
\item[\vocab{reduce/reduce conflicts}] where the parser has seen the entirety of the body of two productions that appear to be valid at this point in the parse and is unable to determine which to reduce to.
\end{description}

\paragraph{LALR$(k)$ Parsers}
More sophisticated parsing methods are more discriminating about what productions are still valid at a given point in the parse by taking into account more or less of the parsing process and input seen thus far, so called \vocab{left context} as it is to the left of where the parser presently is in consuming the input. (In this analogy, the lookahead symbols could be considered right context, though that term is never used.) One such method is known as \vocab{look-ahead LR (LALR)}. These parsers can be seen as ``compressed LR parsers,'' though this compression can introduce spurious reduce/reduce conflicts that would not occur in a full LR parser. This has historically been seen as an acceptable tradeoff for the reduction in table size and construction time, since any LR grammar can be reformulated as an LALR grammar, but with more sophisticated LR algorithms developed later that retained the full power of full LR parsers while producing comparable levels of compression wherever possible (meaning that parsing an LALR grammar with such an LR parser would require the same space as parsing it with an LALR grammar), such a tradeoff became unnecessary, though it remains widespread. 

%TODO: Mention that similar methods allow creating a table-driven implementation of a DFA lexer.

\paragraph{Table-Driven Parsers}
Whereas recursive descent parsers and operator descent parsers can be hand-coded, many of the other parsing algorithms were developed to operate by way of precomputed tables.\footnote{That is not to say that the others cannot also be implemented through tables, simply that the table method is not felt to be the necessity that it is for these others.} They explicitly model a \FA, called the \vocab{characteristic finite automaton}; the tables allow the transition function to be implemented purely by table lookup. As hand-creation of tables is time-consuming and error-prone, tables for parsing are generally created algorithmically and the resulting tables used with a \vocab{driver} that simply does little more than gather the information necessary to perform the operations specified by the table.

\paragraph{Direct-Coded Parsers}
Parsers implemented entirely in code (rather than as a set of tables with a driver) were long seen as something to be generated only by humans, while parsers generated from a higher-level grammar description were to be implemented by way of tables. However, another possibility, often faster and smaller because of its lower overhead and its lack of a need to encode a rather sparse table, is to have the parser generator create a direct-coded parser, a parser that is not table-driven but yet is generated from a higher-level description rather than being written by hand.

\paragraph{GLR Parsers} LR parsers are restricted to parsing only LR languages. However, a very similar technique can be used to parse all context-free languages. \vocab{Generalized LR (GLR) parsers} are more general than LR parsers in two senses: 
\begin{itemize}
\item They are able to parse all context-free grammars, not just LR grammars.
\item Their method of parsing is a generalization of that used in LR parsers.
\end{itemize}
They generalize the parsing method of shift-reduce LR parsers by coping with ambiguity in the grammar by duplicating the parse stack and pursuing competing parses in parallel. When they determine a particular parse is in fact invalid, it and its stack are destroyed. If the grammar is in fact ambiguous and multiple parses are possible, this might lead to a \vocab{parse forest} instead of a parse tree. Making such parsers feasible requires some effort, and part of that effort was to replace several duplicate parse stacks by what amounts to a ``parse lattice'' that share as many grammar symbols as possible as parses converge and diverge, much reducing the space requirements of the parser as well as time spent repeating the identical shifts and reduces on different parse stacks. It is also important to employ similar compression methods as with the newer LR parser generation algorithms, so that extra space and time is only employed as strictly necessary to deal with non-LR constructs.%
%FIXME: Henckell wants an example of a non-LR construct. Would be a good idea. See Elkhound TR; hopefully it gives one.

\paragraph{Semantic Actions}
We generally desire to know more than that a given input is grammatical: we want to create a representation of the information discovered during parsing for later use. This is done by attaching \vocab{semantic actions}, to productions in parsers and to recognized tokens in lexers. Such actions are invoked when the production is reduced or the token recognized, and they are used to build the representation and, in the lexer, to emit the recognized token for the parser's use. They also can be used to compute attributes of the nodes in the parse tree, as discussed next.
%TODO: talk about scannerless parsers -- from incestuous commingling to rad idea -- in the conclusion/bibliography. From modularity to power and effectiveness....


\subsection{Semantic Analysis}
\vocab[semantic analysis]{Semantic analysis}, also known as \vocab{context-sensitive analysis}, follows scanning and parsing. Its job is to ``understand'' the program as parsed. Not all elements of the language can be checked by either regular expressions or context-free grammars; checking these falls to the semantic analyzer. Approaches to semantic analysis vary widely; while a formalism that permits generating semantic analyzers from a higher-level description, as is done for lexers and parsers, exists, its use has yet to become widespread. Frequently, semantic analysis is done purely through \foreign{ad hoc} methods.

A program in truth has two aspects to its semantics, the static and the dynamic. \vocab[static semantics]{Static semantics} are those aspects of the program's meaning that are fixed and unchanging. A common example is the type of variables (though there are languages that employ dynamic typing). These aspects are particularly amenable to analysis by the compiler, and information derived from understanding them can be used to optimize the program. A program's \vocab{dynamic semantics} are those aspects of the program that are only determined at runtime. A compiler can attempt to prove through analysis certain properties of the running program, for example, that an attempt to access an array element that does not exist (the eleventh element of a ten-element array, for example) can never occur. Some languages require that the compiler guarantee certain runtime behavior: if it is unable to provide that guarantee at compile time through analysis, the compiler must insert code to check the property at runtime. Java requires that no out-of-bounds array access occur: any such attempt must be refused and raise an error. Since these runtime checks can slow down a program, a frequent point of optimization in languages requiring such checks is proving at compile-time properties that enable the omission of as many such checks as possible from runtime. Many languages, particularly older languages, do not require runtime checks even where they might be worthwhile, while some compilers might permit disabling the insertion of runtime checks, an option favored by some for the generation of final, production code after all debugging has occurred.

\subsubsection{Attribute Grammars}
The formalism mentioned above for performing semantic analysis is that of \vocab{attribute grammars}. Attribute grammars piggy-back on the concepts of context-free grammars and parse trees. They associate to each grammar symbol a finite set of \vocab{attributes} that store information and to each production \vocab{semantic rules} that specify how the attributes of the symbols involved in that production are to be computed. Attributes are partitioned into two sets, those of the heads of productions, called \vocab{synthesized attributes}, and those of symbols of the body of a production, called \vocab{inherited attributes}. These can be viewed as flows of information respectively up and down the parse tree.

Attribute grammars are used like so: Once a parse tree is constructed, its symbols are decorated with \vocab{attribute instances}. Each symbol has its attributes, and each occurrence of that symbol in the tree has its own instances of those attributes; the attributes are common between two occurrences of the same symbol, but the values of their instances of those attributes will likely differ. A parse tree together with its attribute instances is called an \vocab{attributed tree}.

%CRIT-TODO: Need an example of an attributed grammar here.

It is within this attributed tree that attribute evaluation occurs. \vocab[attribute evaluation]{Attribute evaluation} is the computation of the values of the attribute instances of an attributed tree. Such an evaluation will not necessarily terminate, and determining an appropriate order for evaluation such that evaluation can be performed efficiently is nontrivial.%
%CRIT-TODO: Need to introduce an example of nonterminating evaluation/hard to determine order of evaluation giving termination.

Part of making this formalism usable involves, as with context-free grammars, finding sufficiently powerful, restricted classes of attribute grammars that can be used to capture the semantic information desired while enabling efficient evaluation. Two such classes are the \vocab{S-attributed grammars}, which admit only synthesized attributes, and so can be evaluated through a simple bottom-up walk of the parse tree, and the \vocab{L-attributed grammars}, which permit the attribute values of a given symbol in a production to depend only on the inherited attributes of the head of the production and the synthesized attributes of any symbols to the symbol's left; like the $S$-attributed grammars, they admit information flow from bottom-to-top within the parse tree, but they also allow for left-to-right information flow, as well, and can be readily evaluated in a left-to-right, depth-first walk of the parse tree, as occurs during recursive descent parsing.

Problems faced by practical implementations of the attribute grammar formalism include the management of storage for the multitude of attribute instances used during evaluation and the amount of attributes that exist solely to share non-local information. Non-local information is in general a problem with attribute grammars, and while a symbol table can be used alongside the grammar to avoid this issue, it is also an end-run around the formalism.

%TODO: Move to conclusion of chapter: Attribute grammars are useful to far more than compiler writers; they can be put to good use in the generation of debuggers, syntax-aware editors, and, more broadly, interactive development environments.
