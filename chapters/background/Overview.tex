%\begin{flushright}
% Perhaps this should begin the meat of the text, or, more likely,
% it does not belong at all.
%\begin{verse}
%`The time has come', the Walrus said,\\
%`to talk of many things: \\
%Of shoes---and ships---and sealing-wax---\\
%Of cabbages---and kings---\\
%And why the sea is boiling hot---\\
%And whether pigs have wings.'
%\end{verse}
%\defcitealias
%\end{flushright}

\myChapter{Overview}\label{background:overview}
Before we can discuss compiling functional languages, we must set the scene. 
\begin{description}
\item[\nameref{background:beginnings}] looks into where the imperative and functional paradigms began.
\item[\nameref{background:computers}] outlines the structure of modern computers with an emphasis on those features that particularly affect the design of compilers.
\item[\nameref{background:compilers}] introduces compilers% at a high level
, including their architecture and associated theory, and concludes with a discussion of bootstrapping a compiler.
\end{description}