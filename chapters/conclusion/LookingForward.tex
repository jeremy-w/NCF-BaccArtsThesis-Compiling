\myChapter{Looking Forward}\label{conclusion:forward}
Functional languages are growing up. They are beginning to see increasing use in industry and increasing interest among programmers. They also hold out promise as a way to deal with the rise of ubiquitous symmetric multiprocessors, which brings the problems of concurrent programming out of scientific and network programming and into programming in general.

Functional languages also continue to influence imperative languages. Java brought garbage collection into the mainstream. Several imperative languages, including Microsoft's C\#, now allow anonymous functions; the programming language Python borrowed list comprehensions from Haskell; the spirit of declarative programming, if not explicitly functional programming, shows through in the language-integrated query (\abbrev{LINQ}) facilities added to the .NET platform. mention Scala

fctl prog as a way to improve programming productivity

Have seen how in some ways they are merging, but they continue to develop in their own peculiar ways. Aspect--Oriented Programming; OTT; dependent typing; will logic and constraint-based programming come into its own?

\section{Bibliographic Notes}
use in industry: OCaml funded by, Haskell history, Wadler's and Appel's lists

erik meijer: LINQ and confessions