\myChapter{Languages}\label{functional:languages}
\section{Predecessors}
\subsection{Lisp}

\subsection{Iswim}

\subsection{APL and FP}

\section{Modern Functional Languages}
\subsection{ML}

\subsection{Haskell}

\section{Criticisms}
\subsection{Trivial Update Problem}
\subsection{Explicit versus Implicit Parameters}
\subsection{Input--Output}
\subsection{Lack of Familiarity}

\section{Bibliographic Notes}
\citet{Hudak:Conception:1989} develops the theory associated with the \lambdacalc alongside the history of functional programming. \citet{Hudak:A-history:2007} thoroughly describes the development of Haskell.

Monads, which we introduced in the context of input--output, are used extensively in Haskell but the concept and its uses continue to prove difficult to grasp. Most texts discussing Haskell take a stab at explaining monads along the way, but \citet{Wadler:Monads:1995} focuses exclusively on the use of monads in functional programming. \citet{Peyton-Jones:Tackling:2000} focuses on how Haskell handles the ``awkward squad'' of input--output, concurrency, exceptions, and interfacing with other languages. These topics provide many examples of monads and their uses in Haskell, as well as being of interest for functional languages in general as an record of one set of solutions to some thorny issues for functional programming.

\citet{Scott:Programming:2006} discusses the trivial update problem after introducing functional programming using Scheme (an influential Lisp) and Haskell.