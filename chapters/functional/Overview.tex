\myChapter{Overview}\label{functional:overview}
\partandnameref{Chapter}{background:beginnings}, introduced the fundamentally different concept of computation at the root of functional programming languages as opposed to that at the root of imperative languages. We have discussed imperative languages at some length. Now, we move on to functional languages.

\begin{description}
\item[\nameref{functional:theory}] discusses some theory basic to functional programming.
\item[\nameref{functional:languages}] sketches the history of the functional language family by way of several of its defining languages.
\item[We next turn] to the question of how programs in modern functional languages are compiled to run on modern computers. We will answer this by studying two widely-used compilers:
\begin{itemize}
\item \nameref{functional:ghc} examines the architecture of the principal compiler for the Haskell language.
\item \nameref{functional:ocaml} looks at the compiler provided by the developers of the Objective Caml language.
\end{itemize}
\end{description}