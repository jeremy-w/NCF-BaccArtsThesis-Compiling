\myChapter{Overview}\label{functional:overview}
We have discussed imperative languages at some length. Now, we move on to functional languages.

\begin{description}
\item[\nameref{functional:theory}] discusses some theory basic to functional programming.
\item[\nameref{functional:history}] sketches the history of the functional language family by way of several of its defining languages.
\item[\nameref{functional:compiling}] describes in broad terms how functional languages are compiled. %and introduces the concept of an abstract machine.
\item[\nameref{casestudies}] addresses the question of how programs in modern functional languages are actually compiled to run on modern computers. We answer this by studying two widely-used compilers:
\begin{itemize}
\item The Glasgow Haskell compiler, the principal compiler for the Haskell language.
\item The Standard \ML of New Jersey compiler, an optimizing compiler for the language Standard \ML{}.
\end{itemize}
\end{description}