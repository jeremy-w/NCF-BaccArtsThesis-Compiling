\myChapter{History}\label{functional:history}
\section{Predecessors}
There does not appear to be a consensus on which language was the first, truly functional language. This is because the argument inevitably ends up being about how the terms of the argument should be defined. What is a functional language? Are there elements it must have? Elements it must not? Do research languages count, or does a language have to have seen significant ``real world'' use? Is a given ``language'' really a language, or is it simply a dialect?

You will have to make up your own mind about these matters, possibly on a case-to-case basis. Regardless of your decisions, there \emph{is} a good consensus on which languages contributed to the development of the functional family, regardless of whether or not they truly belong to it. To dodge the whole issue, we will simply characerize them as the predecessors of modern functional languages.

\subsection{Lisp}
The earliest predecessor is the list processing language. Known as LISP when it first appeared in the late 1950s (it was all the rage then to capitalize the names of programming languages), and since grown into a diverse family of Lisps, it appeared shortly after Fortran. It originated with McCarthy, and, in fact, elements of its list processing facilities were first implemented as extensions of Fortran \citep{Stoyan:Early:1984}.

Lisp grew out of artificial intelligence, particularly the expert systems and their need to perform list processing and limited theorem proving. In fact, the list is its primary and most characteristic data type. Lisp programs themselves can be characterized and represented as lists, and this lends Lisp its most distinctive feature: its heavy use of parenthesization. The ability of Lisp to represent itself in itself\empause Lisp lists are Lisp programs are Lisp lists\empause is known as \vocab{homoiconicity}, and this lends Lisp much of its power and extensibility.

This focus on lists is unlike the \lambdacalc{}, which features functions as its sole data type, and even in its extensions remains solidly anchored by its focus on the function. McCarthy draws on the \lambdacalc solely to provide a notation for ``functional forms'' as opposed to functions\empause basically, to indicate which positional argument should be bound to which variable name in a function's definition. In introducing Lisp, he in fact states that the ``$\lambda$-notation is inadequate for naming functions defined recursively'' and introduces an alternate notation \citep{McCarthy:Recursive:1960}. Many languages today get by using the lambda term $Y$ that we introduced earlier for this purpose; the impact of the \lambdacalc on Lisp was superficial, and this is in good part why one might want to exclude Lisp from a list of functional languages.

Much of the spirit of functional languages, however, first appeared in Lisp: functions as ``first-class citizens'' and the use of recursive functions as opposed to step-variable--based loops, as well as an elegant, remarkably simple definition characterize both Lisp and the modern functional languages. As far as elegance goes, it is possible to write a Lisp interpreter in not very many lines of Lisp.

Lisp flourished as artificial intelligence flourished, and it weathered the cold \abbrev{AI} winter, perhaps even better than \abbrev{AI} did itself. It was readily implemented by many groups and extended in many different directions, so Lisp soon became more a family of languages than a single language. From the 1960s on, there were two major Lisps (Interlisp and MacLisp) and many other significant Lisps. Today, the two primary Lisps are Common Lisp, the result of a standardization effort in the 1980s, and Scheme, which in the mid-1970s sprang out of ongoing research in programming language theory and so was inspired more immediately by the \lambdacalc{}.

\subsection{Iswim}
Iswim (for ``if you see what I mean'') is a family of programming languages developed in the mid-1960s by Peter Landin. It is the first language that really looks like modern functional languages. In stark contrast to Lisp, it is not all about lists, it uses infix notation, and it is thoroughly based on the \lambdacalc{}. It also features \code{let} and \code{where} clauses for creating definitions local to a given scope. This is one of the most immediately visually distinctive elements of modern functional languages. Iswim also allowed the use of indentation (significant whitespace) for scoping alongside the more common punctuation-based delimiters.

\subsection{APL and FP}
APL (``a programming language''), developed in the early 1960s by Kenneth Iverson, was never intended to be a functional programming language, but rather an array programming language. Thus, it provided built-in support for operating on arrays in terms of themselves rather than in terms of their elements, as well as ways of composing these array operations. It is also notable for its concision: it was intended to be programmed in using a specialized alphabet. This, coupled with its approach to handling arrays, led to very compact programs.

It appears to have influenced John Backus in his development of FP. FP itself never saw much, if any, use. It was advocated in Backus' 1978 Turing award lecture in which he warned of the ``von Neumann bottleneck'' that ultimately constrains imperative programming languages to ``word-at-a-time programming.'' FP intended to do for functional programming what structured programming did for imperative programming with its standard control flow constructs by providing a few higher-order functions (``functional forms'') that he considered essential and sufficient for whatever one might want to do.

FP is most notable in the history of functional languages for the credibility it lent to the field\empause Backus received the Turing award in good part because of his fundamental role in the development of Fortran\empause and the interest it generated in functional programming. While the development of modern functional programming languages took a different road than that defined by FP, FP's emphasis on algebraic reasoning and programming using higher-order functions is very much of the same spirit.
\section{Modern Functional Languages}
While the who's in, who's out of older languages is up for debate, most modern functional languages bear a close family resemblance. The central features of a modern functional language are:
\begin{itemize}
\item
first-class functions and a firm basis in the \lambdacalc;
\item
static typing coupled with type inference and polymorphism;
\item
algebraic data types and pattern matching.
\end{itemize}
Most modern functional languages also feature:
\begin{itemize}
\item
abstract data types and modules;
\item
equational function definitions and boolean guards.
\end{itemize}
We will discuss each of these in turn.

\subsection{Central Features}
\subsubsection{First-Class Functions and the \LambdaCalc}
It is quite easy to represent functions in the \lambdacalc and to create functions of functions. Such \vocab{higher-order functions} are unusual in imperative languages. Among the provided data types, they are usually second-class citizens: they have no literal representation, but can only be created through statements, nor can they be assigned to variables, passed into or returned from other functions. They are not on par with the integers or even characters.

Functional languages make functions first-class citizens. This means that:\footnote{This particular list is due to~\citet{Mody:Functional:1992}; others provide similar lists of ``rights'' characteristic of first-class data types. Some authors go further and describe the rights typical of second- and third-class data types, as well~\citep[for example][\S 3.5.2]{Scott:Programming:2006}.}
\begin{itemize}
\item
Functions are denotable values: there is some way to describe a function literally, just as you would right \code{5} to denote the integer five without having to give it a name.

\item
Functions can be passed into functions: such functions with functional arguments are known as \vocab[higher-order function]{higher-order functions}.

\item
Functions can be returned from functions.

\item
Functions can be stored in data structures: you can create lists of functions as readily as you would lists of integers.

\item
Storage for functions is managed by the system.
\end{itemize}
With functions as first-class citizens, it easy to create and employ higher-order functions, and functional programming has a rich vocabulary describing common, heavily-used higher-order functions and common types of higher-order functions. First-class functions are also employed extensively in the form of curried functions.

First-class functions is the most striking result of functional languages' basis in the \lambdacalc{}, and it heavily influences the entire style of programming in functional languages. But taking the \lambdacalc as the starting point of the entire programming language is the most radical characteristic of modern functional languages, and the effects of this choice are felt throughout the resulting languages.

\subsubsection[Static Typing, Type Inference, Polymorphism]{Static Typing, Type Inference, and Polymorphism}
Modern functional languages are statically typed. They are based, not on the untyped \lambdacalc{}, but on some variety of the typed \lambdacalc{}. The introduction of types has advantages from the software engineering point of view. It also has advantages from the point of view of compiler performance.

Static typing in imperative languages is often regarded as a burden because of the need to declare the type of all variables and functions. Modern functional languages relieve this burden through type inference. This means that code written in functional languages is free to omit redundant type declarations: if you state that \lstinline{x = 5}, then there is no need to reiterate that \lstinline{x} is an \lstinline{Integer} for the sole benefit of the compiler. Modern functional languages are designed to allow type inference, and their compilers are designed to perform it.

A surprising result of type inference is that it makes polymorphism the standard behavior for functions. Whenever a function could be construed as taking operands of a more general type, it is, unless an explicit type declaration is supplied that restricts this.

The standard higher-order function \lstinline{map} is a good example of this. \lstinline{map} takes as its arguments a function and a list and produces a list containing the results of applying the function to each element of the original list in order. That description is somewhat complex; an example would perhaps be simpler. If we take for granted a boolean function \lstinline{isNonZero} that takes an integer argument and returns either \lstinline{True} if the number is nonzero or \lstinline{False} if it is zero,\footnote{We can define \lstinline{isNonZero} in Haskell as \lstinline{isNonZero x = not (x == 0)}.} then 
\begin{lstlisting}
map isNonZero [0, 1, 2, 3]
\end{lstlisting}
evaluates to 
\begin{lstlisting}
[isNonZero 0, isNonZero 1, isNonZero 2, isNonZero 3]
\end{lstlisting}
and thence to 
\begin{lstlisting}
[False, True, True, True]
\end{lstlisting}
The type of map is \lstinline{(a -> b) -> [a] -> [b]}, where \lstinline{a} and \lstinline{b} here are type variables as discussed in \nameref{types:polymorphism},~p.~\pageref{types:polymorphism}.

\subsubsection{Algebraic Data Types and Pattern Matching}
A distinctive characteristic of the type systems of modern functional languages is their support for creating and using \vocab{algebraic data types (ADTs)}. Algebraic data types are so called because they can be looked upon as a sum of products of other data types. What this means practically is that algebraic data types function as discriminated (tagged) unions; the tags are called \vocab{data constructors} and serve to wrap the supplied data in the algebraic data type. Pairs and lists are simple examples, but since special syntax is often supplied to make their use more natural, they are not very good examples of creating \abbrev{ADT}s.

Let us instead consider an algebraic data type representing a tree with values of some unspecified type stored in the leaves. The declaration of such a data type might look like \begin{lstlisting}
data Tree a = Leaf a | Branch (Tree a) (Tree a)
\end{lstlisting}
(The unspecified type that is being wrapped is represented here as \lstinline{a}.) This also happens to be a \emph{recursive} data type: each branch wraps a pair of subtrees. The declaration \lstinline{tree = Branch (Leaf 1) (Leaf 2)} gives the variable \lstinline{tree} the value of a branch with two leaves of integers. Thus, we have values of type \lstinline{Integer} substituting for the type variable \lstinline{a} in \lstinline{Tree a}. The variable \lstinline{tree} thus has type \mbox{\lstinline{Tree Integer}}, read ``tree of integer,'' and corresponds to the tree
\[
\begin{xy}\SelectTips{eu}{10}
\xymatrix{%
&*{\bullet}\ar[dl] \ar[dr]\\
*++\txt{\code{1}}&&*++\txt{\code{2}}}
\end{xy}
\]

We have seen that it is simple to create an algebraic type and build instances of that type. But how do we get at the wrapped information? To decompose algebraic data types, modern functional languages support \vocab{pattern matching}.

The fundamental pattern-matching construct is the \vocab{case} expression. Its basic form indicates the variable for which cases are being enumerated and then sets up a correspondence between patterns and expressions to evaluate as the value of the \asword{case} expression in the event the corresponding pattern matches the provided variable. The patterns are checked in the order they are listed; the first matching pattern decides which expression is evaluated. Still informally, but somewhat more symbolically, we could represent the form of the \asword{case} expression as
\[
\text{\code{case} } \langle\text{\textit{variable}}\rangle \text{ \code{of} } \left (\langle\text{\textit{pattern}}\rangle \text{ \code{->} } \langle\text{\textit{expression}}\rangle \right)^{+}
\]

As an example, let us suppose we wish to count the number of branches in a tree. An example of such a function is \lstinline{countBranches} of Listing~\ref{features:case} on page~\pageref{features:case}. The patterns are analogous to the expressions we would use to construct the type of data that the pattern matches; the variables of the patterns, rather than passing data into the constructors, instead are used as names for the data that was initially supplied as parameters to the constructor.

The type of this function together with its name provide an excellent summary of its behavior. It also provides us with another example of polymorphism and our first example of subtyping. The type of \lstinline{countBranches} is \lstinline{(Num t1) => Tree t -> t1}. Here, \lstinline{(Num t1) =>} expresses a restriction on the type of the type variable \lstinline{t1} used in the rest of the type expression. It says that the type of \lstinline{t1} must be some subtype of the type class \lstinline{Num}. The underscore you see in the definition of this function is used in patterns as a ``don't care'' symbol: it indicates the presence of a value that we choose not to bind to a name, since we do not intend to refer to the value.

\begin{lstlisting}[float,caption={Pattern-matching via \code{case}},label={features:case}]
countBranches tree = case tree of
                        Leaf   _   -> 0
                        Branch a b -> 1 + countBranches a
                                        + countBranches b
\end{lstlisting}

\subsection{Other Features}
\subsubsection{Abstract Data Types and Modules}
\vocab{abstract data types} are data types that hide their concrete representation from the user. In this way, the representation of the type becomes internal to it: the fact that, say, a stack is actually implemented as a list is hidden, and only operations dealing with stacks as stacks are exposed. This means that the implementation of the abstract type can be changed as necessary. For example, if lists proved too slow to support the heavy use we wished to make of stacks, we could move instead to some other representation without having to change any of the code that used our stacks.

This kind of implementation hiding together with interface exposure is frequently accomplished through a module system. The existence of a powerful and usable module system is an important part of the ``coming of age'' of functional languages, because modules are necessary to support ``programming in the large'' as is necessary in real-world environments where complex problems must be solved and large amounts of code are involved. In terms of modules, an abstract data type's representation is hidden by not exporting representation-specific definitions for use in the program importing the module.

In the context of abstractions of algebraic data types, this takes the form of not exporting the data constructors. Instead, other functions are exported that make use of the data constructors without exposing this fact to the user of the abstract data type. A simple version of such a function would simply duplicate the data constructor. More complex versions can build in bounds-checking, type-checking, or normalization of the representation\empause for example, such a ``smart constructor'' could be used to ensure an internal tree representation remains balanced.

\subsubsection{Equations and Guards}
Modern functional languages support a very readable, compact notation for defining functions that builds on the pattern matching performed by \asword{case} statements. They allow functions to be defined as a sequence of equations. Listing~\ref{features:equations} on page~\pageref{features:equations} reimplements the functionality of Listing~\ref{features:case} (p.~\pageref{features:case}) using an equational style of function definition. If you compare this new definition to the earlier definition, which used the \asword{case} expression, you will see that the pattern matching is implicit in the syntax used to define functions equationally.

\begin{lstlisting}[float,caption={Pattern-matching via equational function definition},label={features:equations}]
countBranches2 (Leaf   _  ) = 0
countBranches2 (Branch a b) = 1 + countBranches a
                                + countBranches b
\end{lstlisting}


Another feature of modern functional languages that simplifies function definition is \vocab{guards}. Guards are boolean predicates that can be used in function definitions and \asword{case} statements. Guards block the expression they precede from being used when they evaluate to false, even if the pattern preceding the guard matches. The first pattern and guard successfully passed determines the case that applies to the given value.

An an example, we could use one of the \lstinline{countBranches} functions given earlier to define an \lstinline{isLeaf} predicate for use with our trees. If the tree has zero branches, it must be a leaf. If it has one or more branches, it must not be. In Listing~\ref{features:guards} on page~\pageref{features:guards}, we use a guard that applies this number-of-branches test in order to prevent the function \lstinline{isLeaf} from evaluating to \lstinline{True} when its argument \lstinline{t} is a tree with more than zero branches.
\begin{lstlisting}[float,caption={Cases with guards},label={features:guards}]
isLeaf t | countBranches t > 0 = False
         | otherwise           = True
\end{lstlisting}

We can also describe guards in terms of how the same effect could be accomplished using other expressions. Guards used with function definitions can be seen as equivalent to chained \asword{if} expressions where each successive guard appears in the \asword{else} branch of the preceding guard.\footnote{We do indeed mean \asword{if} \emph{expressions}, not \asword{if} statements. An \asword{if} expression can be used anywhere an expression is expected. The \asword{else} branch is always required, which means the expression will always have some value, either that of the true or the false branch.} The expressions being guarded in the function definition become the contents of the \asword{then} branch that is evaluated if their guard evaluates to true. A translation along these lines of the \lstinline{isLeaf} function of Listing~\ref{features:guards} (p.~\pageref{features:guards}) is:
\begin{lstlisting}
isLeaf2 t = if branches t == 0 then True else False
\end{lstlisting}
But, since the \lstinline{otherwise} of Listing~\ref{features:guards}~(p.~\pageref{features:guards}) is simply another name for \lstinline{True}, we can produce a more faithful (and redundant) translation of the original \lstinline{isLeaf} function, as shown in Listing~\ref{guards:ifthenelse} on page~\pageref{guards:ifthenelse}.

\begin{lstlisting}[float,caption={Guards as chained if-then-else--expressions},label={guards:ifthenelse}]
isLeaf3 t = if countBranches t == 0
               then True
               else if True 
                       then False  
                       else False
\end{lstlisting}

We can similarly transform a \asword{case} statement that uses both patterns and guards, but this requires a significant amount of nesting and duplication. We must first attempt to match the patterns. As before, if a pattern does not match, the next pattern is tried. Each guard migrates to the corresponding expression. The original expression is wrapped in an \asword{if} expression that tests the corresponding guard condition. If the test succeeds, the \asword{then} branch is the expression corresponding to the pattern just matched and guard just passed is evaluated. Otherwise, we must duplicate the remaining patterns and guards, and transform them similarly.

An example should clarify this. We will not use descriptive names as before, because they would obscure the transformation and motivating such descriptive names would unduly prolong this discussion. The case expression with three guarded branches of Listing~\ref{cases:guards}~(p.~\pageref{cases:guards}) can be transformed as described above into the nested case expressions without guards of Listing~\ref{cases:noguards}~(p.~\pageref{cases:noguards}).

These transformations can be adapted to handle multiple guarded expressions per pattern, but transformation process only becomes more tedious. The examples given should suffice to demonstrate how much the use of guards simplifies both reading and writing of functional programs using pattern matching.

\begin{figure}
\myfloatalign%
\caption[\asword{Case} without guards]{Transforming a \asword{case} statement to eliminate guards}%
\captionsetup{position=top}
\subfloat[With Guards]{%
\label{cases:guards}%
\begin{minipage}{0.4\textwidth}
\begin{tabbing}
\code{case} \=$E$ \code{of} \+\\
    $p_1 \mid g_1 \to e_1$\\
    $p_2 \mid g_2 \to e_2$\\
    $p_3 \mid g_3 \to e_3$
\end{tabbing}
\end{minipage}
}%
\qquad%
\subfloat[Without Guards]{%
\label{cases:noguards}%
\begin{minipage}{0.4\textwidth}
\begin{tabbing}%NOTE: If \code{} changes, so must this!
% Tab stops:
%  |           |   |   |        |
\code{case} \=$E$ \code{of} \+\\
    $p_1 \to B_1$\\
    $p_2 \to B_2$\\
    $p_3 \to B_3$\-\\
\code{where}\+\\
    $B_1 =$ \code{if} \=$g_1$\+\\
                \code{then} $e_1$\\
                \code{else}\= \+\\
                    \code{case} \=$E$ \code{of}\+\\
                        $p_2 \to B_2$\\
                        $p_3 \to B_3$\-\-\-\\
    $B_2 =$ \code{if} $g_2$\+\\
                \code{then} $e_2$\\
                \code{else} \+\\
                    \code{case} $E$ \code{of}\+\\
                        $p_3 \to B_3$\-\-\-\\
    $B_3 =$ \code{if} $g_3$\+\\
                \code{then} $e_3$\\
                \code{else error }\code{("Patterns not"++}\+\+\\
                                  \code{\ \ \ " exhaustive")}%For some reason, tabstops didn't line it up exactly.
%    \code{errorText} $=$ \code{"Non-exhaustive patterns"}
\end{tabbing}
\end{minipage}%
}%
\end{figure}

\section{Classified by Order of Evaluation}
Functional languages developed along two branches. These branches are distinguished by their evaluation strategy: one branch pursued the applicative order, call-by-value evaluation strategy; the other pursued the normal order, call-by-name evaluation strategy. Languages belonging to the applicative order branch are called \vocab{eager languages} because they eagerly reduce functions and arguments before substituting the argument into the function. Presented with the application $f\, M$, where $f \betared f\prime$ and $M \betared M'$, an eager language will reduce $f\, M$ to $f\prime\, M\prime$ and only then substitute $M\prime$ into $f$. Languages that are part of the normal order branch are called \vocab{lazy languages}, because they delay reducing functions and arguments until absolutely necessary. When a function is applied to an argument, they simply substitute the argument wholesale and proceed with reduction of the resulting lambda term. When a lazy language encounters an application $f\, M$ of $f$ to $M$, it immediately performs the substitution of $M$ into $f$.

The branches have also diverged along the lines of purity and strictness. Eager languages have historically been \vocab{impure}, meaning that they allow side effects of evaluation to affect the state of the program. \vocab[destructive update]{Destructive update} (also known as mutation) is a prime example. Using destructive update, we can sort a list in place simply by mutating its elements into a sorted order. Without destructive update, we would be forced to use the old list to produce a new list, which requires us to allocate space for both the original list and its sorted counterpart.

While destructive update might lead to local improvements in efficiency, it and other impurities destroy \vocab{referential transparency}, since the same expression no longer evaluates to the same value at all times and places in the program. Consider the list \lstinline{ell = [3, 2, 1]}. With this definition, \lstinline{head ell} evaluates to \lstinline{3}. But if we sort it in place, later occurrences of \lstinline{head ell} will evaluate to \lstinline{1}. As you can see, \lstinline{head ell} is no longer always equivalent to \lstinline{head ell}: the reference \lstinline{head ell} is no longer transparent.

Losing referential integrity complicates reasoning about the behavior of the program and the development of any proofs about its behavior. While impurity makes it easier to rely on knowledge of data structures and algorithms gained while using imperative languages, it also undermines one of the strengths of functional programming, that its programs are easier to reason about. The ability to fall back on imperative algorithms also stunts the development of purely functional data structures and algorithms. This is impurity as crutch.

While lazy languages have remained pure, this is in good part due to necessity. Lazy reduction makes it difficult to predict when a particular term will be reduced, and so it is hard to predict when the side effects of a particularly reduction would occur and difficult to ensure they occur when you wish.

The decision between strict and non-strict semantics has also frequently fallen along family lines. Eager languages are almost always strict, by which we mean that the functions of that language default to being strict.\footnote{Whether it is even possible to avoid strictness in a particular case, and the particular methods for doing so where it is possible, will differ from language to language.} If they are going to pursue an applicative order reduction strategy, unless they investigate some sort of concurrent pursuit of several reductions simultaneously, then they will be stuck reducing a divergent argument regardless of whether it would be needed by the function once the substitution of the argument into the function is made. This is the case when functions that ignore their argument are applied to a divergent term: $(\lambda x. \lambda y. y) \Omega \betared \lambda y.y$, but if you attempt to evaluate $\Omega$ prior to substituting it for $x$ in the function, the evaluation will diverge. Lazy languages, on the other hand, will not fall into this trap. Their evaluation strategy makes them non-strict.

The way that laziness forces a language to take the ``high road'' of purity has been referred to as the ``hair shirt of laziness'' \citep{Peyton-Jones:Wearing:2003}. The purity that results from adopting non-strict semantics has a pervasive effect on the entire language. For example, one is forced to discover a functional way to cope with input--output, and computation with infinite data structures becomes feasible. Infinite data structures are usable in a lazy language because, so long as only a finite amount of the structure is demanded, evaluation continues only until that amount has been evaluated.

We have provided some background on the two primary branches of the modern functional family. Now we will briefly summarize their history.

\subsection{Eager Languages}
The most influential eager languages have fallen under the umbrella of the ML family. ML originally began as the metalanguage (hence the name) for the LCF theorem prover project under way at the University of Edinburgh in the early 1970s. It bears a resemblance to Landin's proposal for Iswim. It is a modern functional language, and as such supports the features discussed earlier. Its strong emphasis on type inference was groundbreaking. Type inference was made possible by Milner's rediscovery of a type system earlier described by Damas and Hindley that walked the fine line between a too powerful type system in which type inference is infeasible and an overly restrictive type system.

In the late 1980s, ML was standardized under the name Standard ML. Standard ML is unusual among programming languages in that the entire language it has a formal definition, first published in 1990. Standard ML's support for modules (called structures in Standard ML) is unusually extensive and complex; module signatures (interfaces) can be specified separate from the modules themselves, and it is possible to define functions over modules (such functions are known in ML as functors). A revised edition of the definition was published in 1997. Along with some slight changes to the language, the revision introduced the Standard Basis Library in order to specify a common set of functionality that all conforming Standard ML implementations should provide.

ML's background as a metalanguage for a theorem prover is reflected in its continuing use in programming language research and theorem proving. This research is greatly aided by the published standard: extensions of the language have a solid basis on which to build. But SML was not the only outgrowth of ML.

The Caml languages are another branch of the ML family. This branch has arguably eclipsed Standard ML, particularly in the number of non-research uses to which its languages have been put. Caml was originally an acronym for ``Categorical Abstract Machine Language''; the name has been retained, though the abstract machine has long been abandoned in its implementation. The language began development in 1987 for use in projects of the \foreign{Formel} project at \abbrev{INRIA}; the primary outgrowth of this has been the Coq proof assistant. Because the language was meant for internal use, it was not synchronized with Standard ML, since adhering to a standard meant would make it difficult to adapt the language as needed to suit the problems faced in the group's work.

The start of the 1990s saw the reimplementation of the Caml language. This version of Caml was called Caml Light and featured a bytecode compiler. The interpreter for this bytecode was written in C so as to be easily portable. A bytecode-compiled program can run without changes on any platform to which the interpeter has been ported. Caml Light was promoted as a language for education.

In 1996, Objective Caml made its debut. Objective Caml adds support for object-oriented programming to Caml Light, strong module support, an extensive standard library, and compilation to native code in addition to continuing support for bytecode compilation. In the mid-2000s, Objective Caml became the inspiration for Microsoft's F\# programming language meant to be used with their .NET framework.

The Caml family of languages provides a marked contrast to the Standard ML family. While Standard ML was published as a formal document with clear roots in programming language research, the development of the Caml languages is driven by their continued use for day-to-day programming to support other interests. Standard ML is a single language with many independent implementations. The Caml family, on the other hand, is defined by its provided compiler: whatever the compiler will accept is what the language is at any given time. Thus, a Caml language is defined more by its \vocab{reference implementation} than by any formal document. Objective Caml is not just a language, but a compiler and a host of other tools (such as a preprocessor, profiler and debugger, and tools for performing lexing and parsing) that come together to make up the current version of Objective Caml.

\subsection{Lazy Languages}
Lazy: SASL, KRC, Miranda, (Lazy ML), Haskell

History goes here, details of lexical/syntactic/etc.~and formal definitions go in the chapter on Ghc.

%Need to put in a chapter (parallel history?) on abstract machines, balkanization, interpreters versus compilers, and so on and so forth. Should talk about why we chose Ghc and SML/NJ, as well.


%TODO: Stick in conclusion.
%\section{Criticisms}
%\subsection{Trivial Update Problem}
%\subsection{Explicit versus Implicit Parameters}
%\subsection{Input--Output}
%\subsection{Lack of Familiarity}

\section{Bibliographic Notes}
\citet{Hudak:Conception:1989} surveys the history of functional programming languages through the 1980s. It develops the concepts of the \lambdacalc and its extensions in parallel to the history. This survey particularly influenced the overall shape of our history.

Lisp made its debut in McCarthy's seminal paper ``Recursive Functions of Symbolic Expressions and their Computation by Machine, Part~I''~\citep{McCarthy:Recursive:1960}.\footnote{If you read this paper, you will find that we have fudged some of the technical details of Lisp's description and omitted recounting some significant innovations that were not relevant to the body of functional programming. This was intentional.} There is a good body of literature on the history of Lisp. McCarthy gives a recounting of its early history~\citep{McCarthy:History:1978}. \Citet{Stoyan:Early:1984} covers much the same time period, concluding their history a bit before McCarthy, but where McCarthy's history was based primarily on his recollection, theirs is based on written records. It is very interesting to watch the elements of Lisp gradually fall into place here and there throughout various documents. McCarthy's Lisp retrospective~\citep{McCarthy:LISP:1980} provides a very concise recounting of the most significant innovations and characteristic elements of Lisp. \Citet{Steele-Jr.:The-evolution:1993} gives a fascinating recounting of the tumultuous history of the Lisp family that transpired between the early history as described by McCarthy and Stoyan and the standardization of Common Lisp. \Citet{Layer:Lisp:1991} describes the novel elements of the Lisp programming environment, including some information on Lisp machines, computers that were specially developed to support Lisp and its environment. As for Scheme, its community recently (2008) ratified \textit{The Revised$^{6}$ Report on the Algorithmic Language Scheme}.\footnote{Affectionately known as the \abbrev{R$^{6}$RS}; the \abbrev{R$^{6}$R} part stands for the \textit{Revised Revised \dots Revised Report.} For the report itself as well as details on the process that led to its ratification, see \url{http://www.r6rs.org/}.}

Iswim was introduced by Landin~\citep{Landin:The-next:1966} as language framework meant to support creation of full-featured domain-specific languages. FP was first described in Backus's Turing award lecture~\citep{Backus:Can-programming:1978}. APL is described in a book~\citep{Iverson:A-programming:1962} by its creator, Iverson.

Gordon gives a brief history~\citep{Gordon:From:2000} of the LCF theorem prover project that led to ML and of LCF's successors. The type system and inference algorithm described by Milner~\citep{Milner:A-theory:1978} was also independently developed by Curry~\citep{Curry:Modified:1969} and Hindley~\citep{Hindley:The-principal:1969}. Milner's work was subsequently extended by Damas~\citep{Damas:Principal:1982}. The type inference algorithm is known as both the Hindley--Milner algorithm and the Damas--Milner algorithm and centers around the unification of type variables. The algorithms can also be expressed in terms of generating and subsequently solving a system of constraints~\citep{Pottier:A-modern:2005}. Kuan and MacQueen have described~\citep{Kuan:Efficient:2007} how two compilers, one for Standard ML and one the Objective Caml compiler, have improved the efficiency of the algorithm by ranking type variables.

Standard ML~\citep{Milner:The-Definition:1990,Milner:Commentary:1990,Milner:The-Definition:1997} incorporated a module system developed by MacQueen~\citep{MacQueen:Structure:1981,MacQueen:Modules:1984,MacQueen:Using:1986,MacQueen:A-semantics:1994}. Unlike the language definition itself, part of the documentation of the Standard Basis is available online~(\url{http://sml.sourceforge.net/Basis}) as well as in a book~\citep{Gansner:The-Standard:2002}. The website provides only the formal specification; the book includes tutorials and idioms, as well. An initiative~(\url{http://sml.sourceforge.net/}) is under way to support the development of common tools and test suites and more coordination overall between Standard ML implementors and implementations.

The recollections of a member of the team that developed Caml~\citep{Cousineau:A-brief:1996} provided much of the material for our description of the Caml language family. Information on the current status of the various Caml languages can be found online~(\url{http://caml.inria.fr/}).

Documentation of \abbrev{SASL} and \abbrev{KRC} is sparse. Very little on \abbrev{SASL} was published outside technical reports and user manuals. A later version of the user manual~\citep{Turner:SASL:1976} indicates that \abbrev{SASL} was extended with \abbrev{KRC}'s list comprehensions and support for floating point numbers. Another paper~\citep{Richards:An-overview:1984} introduces the implementation of \abbrev{SASL} at the Austin Research Center, which went by the name \abbrev{ARC SASL}. \abbrev{ARC SASL} also included list comprehensions, though there is no indication of floating point support. \abbrev{KRC}~\citep{Turner:The-semantic:1981} was introduced as part of a paper explaining why functional programming languages are superior to others, where it is described succinctly as ``(non-strict, higher order) recursion equations + set abstraction.''

Miranda was created by Turner in the 1980s~\citep{Turner:Miranda:1985} and heavily influenced the design of Haskell. Miranda can now be freely downloaded for personal or educational use from~\url{http://www.miranda.org.uk}. The history of Haskell, including its use of type classes and monads, is thoroughly described~\citep{Hudak:A-history:2007} by several members of the committee that developed the language. Current information, including an up-to-date version of the published Haskell Report~\citep{Peyton-Jones:Haskell:2003} defining the language, is available online~(\url{http://haskell.org/definition}).

%TODO: Stick in conclusion.
%Monads, which we introduced in the context of input--output, are used extensively in Haskell but the concept and its uses continue to prove difficult to grasp. Most texts discussing Haskell take a stab at explaining monads along the way, but \citet{Wadler:Monads:1995} focuses exclusively on the use of monads in functional programming. \Citet{Peyton-Jones:Tackling:2000} focuses on how Haskell handles the ``awkward squad'' of input--output, concurrency, exceptions, and interfacing with other languages. These topics provide many examples of monads and their uses in Haskell, as well as being of interest for functional languages in general as an record of one set of solutions to some thorny issues for functional programming.
%
%\citet{Scott:Programming:2006} discusses the trivial update problem after introducing functional programming using Scheme (an influential Lisp) and Haskell.