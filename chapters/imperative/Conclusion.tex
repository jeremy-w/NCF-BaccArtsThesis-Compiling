\myChapter{Conclusion}\label{imperative:conclusion}
Imperative languages developed to replace assembly languages for general programming purposes. They have dominated the programming language landscape, and their long history and wide use have made them the target of much research. They provide the conventional backdrop against which other programming language families, such as the functional languages discussed next, play their part. Unconventional ideas are often explained in terms of concepts familiar from the imperative paradigm. Alternative paradigms are judged in light of the successes and failures of the imperative. Thus, in addition to technical background, this part serves to communicate something of a common cultural background, as well.
\begin{itemize}
\item In \partandnameref{Chapter}{imperative:defining}, we 

\item In \partandnameref{Chapter}{imperative:compiling}, we 

\item In \partandnameref{Chapter}{imperative:optimizing}, we 
\end{itemize}

%\section{Bibliographic Notes}\label{imperative:conclusion:notes}
% Is this necessary any more, or do the chapter-ending notes suffice?